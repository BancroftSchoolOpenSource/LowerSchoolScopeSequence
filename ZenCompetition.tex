\documentclass{article}
\usepackage{enumitem,amssymb}
\newlist{todolist}{itemize}{2}
\setlist[todolist]{label=$\square$}
\usepackage{easylist}
\begin{document}
	
{\huge \textbf{Zen Robotics Pentatholon}}

\vspace{1cm}

{\huge \textbf{Purpose}}
\vspace{1cm}

The purpose of this competition is to encourage students to become good engineers. A good engineer is defined as someone who builds systems that accomplish tasks with accessible designs, give and revive aid among peers, and competes to push forward the edge of the known possible. In short, we intend to inject hope into the world through inspiring feats of engineering.

Accessible designs means that the designs, tools and components are accessible to all competitors, and non competitors equally. Designs and code should be housed in publicly facing VCS system. Tools should be free and open source to all people. The intent is to prevent enclosure of students skills by companies.  Components should be generic commodity components or be manufactured with open licenses for multi-vendor sourcing.  

To give and revive aid means that you use the tools and sources of others and assume that the tools and code you create will be used. This means being a responsible user of open source code. You should also be a responsible creator of code and documentation. The aid you revive, and then pay forward should be  balanced. 

To push forward the edge of the known possible means that we compete against physics, not against each other nor against the rules. The purpose of the rules is to ensure collaboration and an even playing field. Since there can be as many winners as there are teams that accomplish the task, there is not a motivation to compete against another team of students. We want to see designs and software that can accomplish tasks, and to see how different folks choose and remix designs to push forward. 

  
This competition is designed to encourage these principals in each student. The intent of the rules and scoring structure is to encourage collaboration across teams and a across the world. Students compete against the competition, not against each other. Each teams accomplishments are intended to be acknowledged on their own merits. We are all competing together against the edge of the known possible, 
\pagebreak

{\huge \textbf{Overview}}
\vspace{1cm}

\textbf{Competition description:} The competition will consist of a standard obstacle course in 5 stages, through which a robot will traverse. Robots will be constrained by the design and construction rules below. Obstacles will be added as needed to the end of the course to ensure balanced difficulty. Early sections of the course will be stable for an extended period of time. 

Stage one is a performance to hype up and interact with the crowd.

Stage two is the first stage of obstacle course. 

Stage three is the midpoint performance.

Stage four is the last portion of obstacle course. 

Stage five is the final performance and celebration. 

Students will compete in independent categories for:
\begin{enumerate}
	
\item distance through the course
 
\item  time of completion
 
\item  design contribution
 
\item  code contribution
 
 \item build quality, adornment and animation
\end{enumerate}

 At the end of the course will be a button, which the robot must press to have completed the course. The time through the course is measured and compared for the completion time award. Separate awards will be given for each robot to accomplish the course. If no robot accomplishes the whole course, then the robots will be scored solely on total distance through the course. 
 
 An award for contribution to code will be given to the team that contributes a code contribution that affects the control of the robot and accomplishes some aspect of the challenge, and is well documented and easily accessible for other teams. 
 
 An award for build quality, adornment and animation will be given for the care given to the aesthetics of the robot and especially the attention paid to human robot interaction. Robots that are perceived well and interact well with judges will be eligible for this award. 
 
 \pagebreak
 
{\huge \textbf{Definitions}}
\begin{enumerate}
	\item \textbf{Open Source:} When all of the files used to create an application or design are available.
	\item \textbf{Vitamins:} components generally available, preferable with multiple sources
	\item \textbf{Save-point:} A point in the course where your robot is expected to perform an animation for the crowd.  
\end{enumerate}

  \pagebreak
{\huge \textbf{Rules}}

\begin{enumerate}
	
\item \textbf{Composition Of Robot:} A robot that competes in the competition must be composed of FFF printed parts made on an approved printer in an approved material with sources provided, Viamins from the approved list with BoM provided and cloth or paper elements with provided sources. Using the posted price list, all vitamins in a submitted robot must not exceed \$500

\item \textbf{Open Source Requirement:} All of the sources for the design elements of the robot, the robots control code, all dependent libraries, and all software tools used must be open source at the time of competition. 

\item \textbf{Program execution:} Robots must communicate to the competition station using the provided protocol. A robots competition code is run by the judges based on the Git URl and tag associated with the competition. 


\item \textbf{Performance:} Before proceeding into the obstacle course, the robot will be judged on a 90 second animation intended to hype up the crowd. At each save-point through the course another 90 second animation will be required. 

\item \textbf{Eligibility:} High school and Jr High school students affiliated with public schools or not for profit organizations. 

 \pagebreak

{\huge \textbf{Competition structure}}

The competition will be held over the same weekend by all teems wishing to compete. 

Teams will register with the completion central ranking server and may submit early if they so choose. Early submissions with novel code elements may be picked up by other teams, increasing the likelihood the team that posts early gets recognition for design contributions award.

All teams that compete in the code portion must have at least on hardware submission. Multiple code teams may use a single hardware entry. A code entry may be used on more than one hardware entry. 

Hardware only teams may compete for the design award. They must demonstrate that the design works with an existing software release, or that they have developed and released a working software stack by performing well in the obstacle course. 

Performance only teams may compete so long as the animations and costume can be demonstrated to still work within the obstacle course. 

All teams that wish to compete for top prizes must compete over the competition weekend. All robots source must be posted and tagged by the provided time the day before competition. 

Competition consists of the judges running the tagged code against the student provided robot, connected to the judges station's network. The judge will run the tagged code in a standard execution environment. Students code must work as a self contained system, or bootstrap components as they are needed. Judges station will have access to the internet during practice run to ensure bootstrapping. Student downloads are limited to 8gb of total used disk space. The student provided software will have a competition.groovy file that will perform then entire task of animation, course navigation and stopping at the end to press the button. Animation audio will play through the station computer. 

Each team in the competition must perform some portion of the obstacle course to be counted for judging. All runs of the course are recorded for the instant runnoff judging. Judging for objective measures such ad distance and timme will be judged instantly. Performance and design contribution judging will be performed over the following weeks. Finalists will be invited to submit a new animation for second round judging at a central event. The finalists will be judged and awarded at that time. 



 

\end{enumerate}
\vspace{1cm}
 
 

\end{document}
